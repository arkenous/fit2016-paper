\section{はじめに}
スマートフォンが普及しつつある現在,スマートフォンにおける個人認証の方法は画面上に表示されるソフトウェアキーボードのテンキーパッドを用いたパスコード認証と指紋認証が大部分を占めている.
しかし,パスコード認証を用いる場合は画面ロックを解除するたびに画面に表示されたソフトウェアキーボードを目で見て指でタッチして操作する必要があり,ユーザにとって煩雑な作業である.
また,パスコードはあらかじめ決められた文字種の中から一つずつ文字を選択し,これを並べて構築していく.
この性質上パスコードのパターン数は限られてしまい,認証に用いる鍵の自由度が制限されてしまう.

指紋認証を用いる場合は,認証を行う際に指紋の読み取りモジュールに指を重ねるだけなのでユーザにかかる負担は比較的軽い.
だが,指紋認証を行うためには指紋を読み取るための専用のハードウェアが必要である.
また,指紋そのものは変更できない.
そのため指紋情報が万が一第三者に漏洩した可能性がある場合,今後はその指を認証に用いることができなくなるという問題点がある.

そこで,本研究では人間の動き(以下,モーション)を用いた個人認証アプリケーションを開発する.
これによりパスコード認証が抱える認証の煩雑さを軽減し,かつ認証に用いる鍵の情報が漏洩した際にも鍵の変更が可能となる.
加えて,モーションを鍵とすることで,自由度が高くより直感的に個人認証が行える.
このアプリケーションには,一般的なスマートフォンに搭載されている加速度センサと角速度センサを用いる.

