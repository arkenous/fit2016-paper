\section{モーションデータの加工}
本システムでは,加速度センサ及び角速度センサから得られたデータに対して複数の加工を施している.
まず,本システムではモーションの入力を任意の長さで行える.
この際,登録モードにおける3回のモーション入力によるモーションデータ,認証モードでは登録されたモーションデータと新たに入力されたモーションデータ間でデータ長の差異が生じる可能性がある.
これによる類似度の低下を防ぐために,登録モードでは最も入力時間の長かったデータを基準に他のデータの末尾にゼロを補填する方法を用いている.
認証モードでは登録されたデータの長さを基準に新たに入力されたモーションデータが短い場合は末尾にゼロを補填し,長い場合は末尾を切り落とす方法でデータ長を揃えている.

データ長を揃えた後,動きの小さいモーションについても類似度を測定できるように,データの振れ幅増幅を行う.
登録モードではあらかじめ設定しておいた増幅器の閾値を元に,取得したデータの振れ幅が下回った場合はモーションの動きが小さいとし,全てのデータに増幅量を掛けることで振れ幅の増幅を行う.
認証モードでは,登録モードにおいてモーションデータとともに登録された増幅量を元に,新たに取得したデータに対して増幅処理を行う.

データの振れ幅増幅の後,フーリエ変換を用いたローパスフィルタ処理によって,モーション入力時の手の震えなどから生じうるデータへの影響を取り除く.
これら処理によって得られた加速度データから距離を,角速度データから角度を求める処理を行っている.

