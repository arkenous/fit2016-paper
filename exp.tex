% @suppress
\section{評価}
本システムの有用性を確認するため,実験を行った.
13人の被験者に協力してもらい,円・三角・寝かせて起こす・顔に近づける・上下に2往復振る,という五つのモーションについてそれぞれ入力してもらった.
登録及び認証の試行回数は3回までとし,この回数内で登録及び認証できた場合に成功とした.
被験者には各モーションのそれぞれについて,まず登録をしてもらい,登録できた場合はすぐに認証をしてもらった.
この実験より得られた結果について,各モーション毎の登録及び認証の成功率をまとめたものを表\ref{motion}に,各ユーザ毎の登録及び認証の成功率をまとめたものを表\ref{user}に示す.

\begin{table}[!htbp]
    \centering
    \caption{各モーション毎の登録及び認証の成功率}
    \begin{tabular}{|c|r|r|} \hline
        & 登録 & 認証 \\ \hline
        circle & 92\% & 91\% \\ \hline
        triangle & 92\%  & 75\% \\ \hline
        lay & 53\% & 100\% \\ \hline
        face & 53\% & 100\% \\ \hline
        shake & 76\% & 90\% \\ \hline
    \end{tabular}
    \label{motion}
\end{table}

\begin{table}[!htbp]
    \centering
    \caption{各ユーザ毎の登録及び認証の成功率}
    \begin{tabular}{|c|r|r|} \hline
        & 登録 & 認証 \\ \hline
        A & 80\% & 100\% \\ \hline
        B & 80\% & 75\% \\ \hline
        C & 100\% & 100\% \\ \hline
        D & 60\% & 66\% \\ \hline
        E & 60\% & 100\% \\ \hline
        F & 100\% & 80\% \\ \hline
        G & 40\% & 100\% \\ \hline
        H & 100\% & 80\% \\ \hline
        I & 60\% & 66\% \\ \hline
        J & 40\% & 100\% \\ \hline
        K & 60\% & 100\% \\ \hline
        L & 100\% & 100\% \\ \hline
        M & 80\% & 100\% \\ \hline
    \end{tabular}
    \label{user}
\end{table}

また,この実験を行うにあたり,登録と認証において被験者がモーションを入力する様子を肩越しに見るような角度でビデオ撮影した.
この撮影データを用い,全てのモーションについて登録及び認証が成功した被験者Cと被験者Lを対象に全てのモーションのなりすまし認証を試行した.

