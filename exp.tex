% @suppress
\section{評価及び考察}
本システムの有用性を確認するため,実験を行った.
13名の被験者に協力してもらい,円・三角・寝かせて起こす・顔に近づける・上下に2往復振る,という五つのモーションについてそれぞれ入力してもらった.
登録及び認証の試行回数は3回までとし,この回数内で登録及び認証できた場合に成功とした.
被験者には各モーションのそれぞれについて,まず登録をしてもらい,登録できた場合はすぐに認証をしてもらった.
この実験より得られた結果について,モーション及びユーザ毎の登録及び認証の成功率をまとめたものを表\ref{test}に示す.

\begin{table}[htbp]
    \centering
    \caption{モーション及びユーザ毎の登録及び認証の成功率}
    \begin{tabular}{c}
        \begin{minipage}{0.46\hsize}
            \centering
            \begin{tabular}{|c|r|r|} \hline
                & 登録 & 認証 \\ \hline
                circle & 92\% & 91\% \\ \hline
                triangle & 92\%  & 75\% \\ \hline
                lay & 53\% & 100\% \\ \hline
                face & 53\% & 100\% \\ \hline
                shake & 76\% & 90\% \\ \hline
            \end{tabular}
        \end{minipage}
        \begin{minipage}{0.46\hsize}
            \centering
            \begin{tabular}{|c|r|r|} \hline
                & 登録 & 認証 \\ \hline
                A & 80\% & 100\% \\ \hline
                B & 80\% & 75\% \\ \hline
                C & 100\% & 100\% \\ \hline
                D & 60\% & 66\% \\ \hline
                E & 60\% & 100\% \\ \hline
                F & 100\% & 80\% \\ \hline
                G & 40\% & 100\% \\ \hline
                H & 100\% & 80\% \\ \hline
                I & 60\% & 66\% \\ \hline
                J & 40\% & 100\% \\ \hline
                K & 60\% & 100\% \\ \hline
                L & 100\% & 100\% \\ \hline
                M & 80\% & 100\% \\ \hline
            \end{tabular}
        \end{minipage}
    \end{tabular}
    \label{test}
\end{table}

また,この実験を行うにあたり,登録と認証において被験者がモーションを入力する様子を肩越しに見るような角度でビデオ撮影した.
全てのモーションについて登録及び認証が成功した被験者Cと被験者Lを対象に,この撮影データを用いて3回までの試行でなりすまし認証が可能か確認した.
得られた結果について,各被験者の各モーション毎に3回の試行で最も高かったコサイン類似度をまとめたものを表\ref{spoof}に示す.

\begin{table}[htbp]
    \centering
    \caption{なりすまし認証でのコサイン類似度}
    \begin{tabular}{|c|r|r|r|r|} \hline
        & \multicolumn{2}{|c|}{C} & \multicolumn{2}{|c|}{L} \\
        & 距離 & 角度 & 距離 & 角度 \\ \hline
        circle & 0.21 & 0.04 & 0.08 & -0.24 \\ \hline
        triangle & 0.11 & -0.19 & -0.12 & -0.13 \\ \hline
        lay & -0.06 & -0.27 & -0.16 & 0.02 \\ \hline
        face & 0.34 & 0.50 & 0.45 & 0.55 \\ \hline
        shake & 0.28 & 0.09 & 0.03 & 0.20 \\ \hline
    \end{tabular}
    \label{spoof}
\end{table}

表\ref{test}から,端末を寝かせて起こすモーションや顔に近づけるモーション,端末を上下に振るモーションといった単純なモーションについて認証が可能であることがわかった.
しかし,全てのモーションについて登録及び認証が成功した被験者が2名しかおらず,成功しやすい被験者と失敗しやすい被験者に二分化する傾向が見られた.
また,被験者から登録や認証に成功するために馴れが必要であるとの声が聞かれた.

表\ref{spoof}から,ほぼ全てのなりすまし認証は失敗したが,被験者Lの顔に近づけるモーションでのなりすまし認証のみ成功する結果となった.
現在のシステムでは,認証の可否を判断する際に距離と角度のコサイン類似度の平均が0.5を上回った場合に認証成功としている.
今回なりすまし認証が成功したケースでは距離についてコサイン類似度が0.45であるにもかかわらず,角度のコサイン類似度が0.55であるために平均すると0.5を越える結果となった.
角度のコサイン類似度が高くても距離のコサイン類似度が低ければ同一のモーションとは考えにくい.
距離と角度それぞれについてコサイン類似度が閾値を越えているかを判断し,これによって認証の可否を判断するように改善する必要がある.

