\section{関連研究}
坂本の研究\cite{sakamoto}では,ユーザが入力したモーションの数値化に加速度センサを用いた.
あらかじめ保存しておいた複数種類のジェスチャパターンと認証時にユーザが入力したモーションデータをパターンマッチング方式のアルゴリズムを用いて比較することで個人認証を行った.
しかし,このプログラムは扱うジェスチャによって認証率が高いものと低いものに二分化する傾向が見られるという問題点があった.

\UTF{6FF5}野らの研究\cite{hamano}では,加速度センサに加えて角速度センサを用いたジェスチャ動作による認証手法を提案した.
これにより回転動作の取得によるモーションの自由度向上となりすまし認証に対する強度の向上を可能にした.
認証手法として単一動作を組み合わせて認証する単一動作組み合わせ認証と,ユーザが自由に考えたモーションを用いてDPマッチングによって認証する一筆書き認証の二つを提案した.
%単一動作組み合わせ認証では,モーション入力より得られた加速度データ・角速度データのそれぞれx,y,z軸の六つのラベルからもっとも分散の大きいものをジェスチャパターンとして識別する.
%識別したジェスチャパターンをパスワードのように組み合わせ,登録されたジェスチャパターン列と一致しているかを判定することで認証した.
%一筆書き認証では,あらかじめ登録されたマスタデータと新たに入力したサンプリングデータをDPマッチングを用いて類似度を比較し,あらかじめ定められた閾値を用いて認証した.
%この研究ではモーションデータに含まれる手振れなどの微小なノイズを除去するために移動平均を施しており,モーション入力毎の時間や速度の差異を抑えるためにデータ長や振幅の正規化を行っている.
%実験より,初日の本人拒否率の平均が一筆書き認証では61\%,単一動作組み合わせ認証では52\%となった.
%また筆者自身がなりすました際の他人受容率の平均が一筆書き認証では12\%,単一動作組み合わせ認証では24\%となった.
このシステムの実証実験は複数日かけて実施されており,一筆書き認証において日を経ることによる習熟度の向上から,本人拒否率が改善したことが確認された.
しかし,初日の認証での本人拒否率が高く,さらなる本人拒否率の改善が課題として挙げられていた.

