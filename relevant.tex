\section{関連研究}
坂本の研究\cite{sakamoto}では,ユーザが入力したモーションの数値化に加速度センサを用いた.
あらかじめ保存しておいた複数種類のジェスチャパターンと認証時にユーザが入力したモーションデータをパターンマッチング方式のアルゴリズムを用いて比較することで個人認証を行った.
しかし,このプログラムは扱うジェスチャによって認証率が高いものと低いものに二分化する傾向が見られるという問題点があった.

兎澤の研究\cite{tozawa}では,ユーザが入力したモーションの数値化にiPodに搭載されている加速度センサと角速度センサを用いた.
モーションの取得は3秒間で30ミリ秒ごとに加速度センサ及び角速度センサからデータを取得する.
ユーザに3回入力させたモーションの平均値データと認証時にユーザが入力したデータの類似性を相関係数を用いて調べることで,個人認証を行った.
これにより,坂本の研究で指摘されていた成功率の二分化や立体的な動きへの対応を可能にし,モーションの対応幅を広げることができたとしている.
しかし,全体的な認証成功率が低く,特に手首のスナップを用いるような動きの小さいモーションに対して認証成功率が特に低く出るなど対応できるモーションに限りがあるという問題点が指摘されていた.

